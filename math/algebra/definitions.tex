\documentclass[11pt, a4paper]{jsarticle}

\begin{document}

\section {距離の公理}

集合$A$の任意の元$x$に対して実数$||x||$が存在し

\begin{itemize}
  \item $ \|x\| \leq 0 $
  \item $ \|x+y\| \leq \|x\| + \|y\| $
  \item $ \|x\| = 0 \Leftrightarrow x = 0 $
  \item $ \|ax\| = \|a\| \|x\| (a \in R) $
\end{itemize}

が成立するとき,$||x||$を\textbf{ノルム(norm)}といい,$A$を\textbf{距離空間(metric space)}という.

\section {行列}

\subsection{行列の積}

$A$が$(l, m)$型で$(i, j)$成分が$a_{i, j}$であるような行列,
$B$が$(m, n)$型で$(i, j)$成分が$b_{i, j}$であるような行列であるとすると,
その積$AB$の$(i, j)$成分は
$$
\sum_{k=1}^{m}a_{ik}b_{kj}
$$

\subsection{逆行列}
$AX = I$かつ$XA = I$を満たす行列$X$を$A$の\textbf{逆行列}といい,$A^{-1}$で表す.

\subsection{正則}
行列$A$が逆行列を持つとき$A$は\textbf{正則(regular)}であるという.

\subsection{対称行列}
正方行列$A$の成分を$(i, j)$成分を$a_{i, j}$とすると,
$a_{i, j} = a_{j, i}$が成り立つとき$A$を\textbf{対称行列(Symmetric Matrix)}という.

\subsection{交代行列}
正方行列$A$の成分を$(i, j)$成分を$a_{i, j}$とすると,
$a_{i, j} = -a_{j, i}$が成り立つとき$A$を\textbf{交代行列(Skew-Symmetric Matrix)}という.

\subsection{直行行列}
実行列$A$が$A^tA = I$を満たすとき$A$を\textbf{直行行列(Orthogonal Matrix)}という.


\end{document}

